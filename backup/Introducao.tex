\chapter{Introdução}
Um problema é dito de otimização combinatória quando envolve escolher uma solução ótima de um conjunto finito de soluções. Neste capítulo serão abordados conceitos importantes sobre grafos, que são importantes na otimização combinatória.

\section{Grafos}
Sejam $V$ e $E$ conjuntos de nós e arestas, respectivamente, o par \(G = (V, E)\) é considerado um grafo. Mais precisamente, \(E\) é um conjunto de pares \(\{i, j\}\) tais que \(i, j \in V, i \neq j\). Em outras palavras, qualquer conjunto de nós e arestas que conectem esses nós pode ser considerado um grafo. Um exemplo de grafo é mostrado na Figura \ref{fig:exemploGrafo}. Nela, \(V = \{1,2,3,4,5\}\) e \(E = \{\{1,2\}, \{2,3\}, \{2,4\}, \{3,5\}\}\).

\begin{figure}[!htbp]
    \centering
    \tikzset{mystyle/.style={circle, draw=black,inner sep=0pt, minimum width=12pt}}

    \begin{tikzpicture}
    
    \node[mystyle] (v1) at (-3,0) {1};
    \node[mystyle] (v2) at (0,2.5) {2};
    \node[mystyle] (v3) at (1,-2.5) {3};
    \node[mystyle] (v4) at (3,2) {4};
    \node[mystyle] (v5) at (4,-0.5) {5};
    \draw[-] (v1)--(v2);
    \draw[-] (v2)--(v3);
    \draw[-] (v5)--(v3);
    \draw[-] (v2)--(v4);

    
    \end{tikzpicture} 
    \caption{Exemplo de grafo não-orientado.}
    \label{fig:exemploGrafo}
\end{figure}

\subsection{Grafo completo}
Se, num grafo \(G = (V, E)\), a condição \(\{i, j\} \in E \;\; \forall i,j \in V, i \neq j\) é satisfeita (todos os nós estão conectados uns aos outros por arestas), \(G\) é um grafo completo.

\subsection{Grafo orientado}
Se, num grafo \(G = (V, E)\), \(\{i, j\} \neq \{j, i\} \;\; \forall i, j \in V, i \neq j\), \(G\) é um grafo orientado (ou dirigido), e \((i, j), (j, i) \in E\) são considerados arcos (e não arestas). Um exemplo de grafo orientado é mostrado na Figura \ref{fig:exemploGrafoOrientado}.

\subsection{Caminho}
Considere um grafo \(G = (V, E)\). Se houver uma sequência de nós \((v_1, \dots, v_n)\), a sequência de arestas \(e_i = \{v_i, v_{i+1}\} \in E, 1 \leq  i \leq n-1\) é um caminho. Observe que tanto a sequência de nós quanto a sequência de arestas podem possuir repetições.

\subsection{Ciclo}
Um ciclo é um caminho que inicia e termina em um nó.

\subsection{Ciclo Hamiltoniano}
Um ciclo Hamiltoniano é um ciclo que visita todos os nós de um grafo uma única vez.

\begin{figure}[!htbp]
    \centering
    \tikzset{mystyle/.style={circle, draw=black,inner sep=0pt, minimum width=12pt}}

    \begin{tikzpicture}
    
    \node[mystyle] (v1) at (-3,0) {1};
    \node[mystyle] (v2) at (0,2.5) {2};
    \node[mystyle] (v3) at (1,-2.5) {3};
    \node[mystyle] (v4) at (3,2) {4};
    \node[mystyle] (v5) at (4,-0.5) {5};
    \draw[->] (v1)--(v2);
    \draw[->] (v2)--(v3);
    \draw[->] (v5)--(v3);
    \draw[->] (v2)--(v4);
    \end{tikzpicture} 
    \caption{Exemplo de grafo orientado.}
    \label{fig:exemploGrafoOrientado}
\end{figure}

\section{TSP}
O Problema do Caixeiro Viajante (\textit{Traveling Salesman Problem} (TSP), em inglês) se trata de um problema de otimização combinatória famoso na literatura. Dado um grafo \(G = (V, E)\), em que \(V\) é um conjunto de cidades e \(c_{ij}\) é o custo de viagem da cidade \(i \in V\) para a cidade \(j \in V\) --- associado às arestas \(\{i, j\} \in E\) ---, o TSP visa  encontrar o ciclo Hamiltoniano (ou rota) que possui o custo de viagem total mínimo.

\subsection{Instâncias do TSP}
Qualquer conjunto de cidades com latitudes e longitudes --- ou coordenadas quaisquer --- pode ser entendido como uma instância do TSP. Em posse dessas coordenadas, é possível calcular a distância entre qualquer par de cidades. É possível, ainda, associar essas distâncias a uma matriz \(M\), na qual o elemento \(m_{ij}\) representa a distância entre a cidade \(i\) e a cidade \(j\).

Na Figura \ref{fig:instanciaTSP}, um grafo completo e não orientado com 4 nós representa uma instância do TSP. Os números próximos às arestas representam as distâncias entre as cidades. A mesma instância pode ser representada pela matriz de distâncias\footnote{Quando a instância se trata de um grafo não orientado --- como no exemplo ---, a matriz \(M\) é simétrica.} \[M = \begin{bmatrix}
0 & 40 & 30 & 20  \\
40 & 0 & 35 & 25  \\
30 & 35 & 0 & 45  \\
20 & 25 & 45 & 0  \\
\end{bmatrix}.\]

\begin{figure}[!htbp]
    \centering
    \tikzset{mystyle/.style={circle, draw=black,inner sep=0pt, minimum width=16pt}}

    \begin{tikzpicture}
    
    \node[mystyle] (v1) at (0,1) {1};
    \node[mystyle] (v2) at (0,5) {2};
    \node[mystyle] (v3) at (6,0) {3};
    \node[mystyle] (v4) at (4,7) {4};
   
   \path
    (v1) edge node [left] {40} (v2)
    (v1) edge node [below] {30} (v3)
    (v1) edge node [below right]{20} (v4)
    
    (v2) edge node [below left] {35} (v3)
    (v2) edge node [above] {25} (v4)

    (v3) edge node [right] {45} (v4);


    
    \end{tikzpicture} 
    \caption{Exemplo de instância.}
    \label{fig:instanciaTSP}
\end{figure}

\section{Meta-heurísticas e algoritmos exatos}
Os métodos heurísticos visam obter boas soluções para problemas de otimização combinatória em pouco tempo. Contudo, soluções ótimas não são garantidas. As meta-heurísticas são procedimentos genéricos que fazem uso de métodos heurísticos para resolver problemas de otimização. Uma mesma meta-heurística pode ser adaptada para resolver diferentes problemas --- levando-se em conta as particularidades de cada um. 

Os algoritmos exatos, por sua vez, garantem a obtenção de soluções ótimas --- embora não consigam fazê-lo em pouco tempo, muitas vezes.

Tanto as meta-heurísticas quanto os algoritmos exatos serão abordados nos próximos capítulos.
